\documentclass [10pt] {article}
\usepackage[cp1251]{inputenc}
\usepackage[english]{babel}
\usepackage{fancyhdr}
\usepackage{graphicx}
\usepackage{times}
\usepackage{amsmath}
\usepackage{longtable}
\usepackage { multicol } 
\usepackage{array}

\setlength { \columnsep } { 1cm }

\setlength{\textwidth}{18cm} 
\setlength{\textheight}{26cm}
\setlength{\topmargin}{-2cm} 
\setlength{\headsep}{0.3cm} \evensidemargin=1cm
\oddsidemargin=-0.8cm

\pagestyle{fancy}
\fancyhead{}
\fancyhead[LO]{\footnotesize\textsf{7920}}
\fancyhead[CO]{\footnotesize\textsf{J. Appl. Phys., Vol. 94, No. 12, 15 December 2003\qquad\qquad\qquad\qquad\qquad\qquad\qquad\qquad\qquad\qquad\qquad\qquad\qquad\qquad\qquad}} 
\fancyhead[RO]{\footnotesize\textsf{Fellah }\textit{et al.} }
\fancyfoot{}
\renewcommand{\headrulewidth}{0pt}
\renewcommand{\baselinestretch}{0.85}
\setlength{\columnsep}{0.7cm}

\begin{document}
\large
\twocolumn
\parindent -1em
\includegraphics[width=1\linewidth]{g1.jpg}

\parindent 0em
{\footnotesize{FIG. 8. Variation of cost function U with the porosity for plastic foam M2.}}

\parindent 1.5em
\textnormal{\linespread{0.5}{\\\normalsize{\indent When the incident angle is $\theta<\theta_c$ , the reflection coefficient
decreases slowly with the incident angle, and when it is
$\theta>\theta_c$ , the reflection coefficient increases quickly with the
angle. It can also be seen from Figs. 2 and 3 that the sensitivity
of porosity variation is more important than the sensitivity
of tortuosity on the reflection coefficient at the first interface.
\\\\
\textbf{V. INVERSE PROBLEM}
\\ [1ex]
\indent The propagation of acoustic waves in a slab of porous
material in the high frequency asymptotic domain is characterized
by four parameters: porosity $\phi$, tortuosity $\alpha_\infty$ , viscous
characteristic length $\Lambda$, and thermal characteristic
length $\Lambda'$, the values of which are crucial to the behavior of
sound waves in such materials. It is of some importance to
work out new experimental methods and efficient tools for
estimation of them. The basic inverse problem associated
with the slab may be stated as follows: from measurement of
the signals transmitted and/or reflected outside the slab, find
the values of the medium�s parameters. The inverse problem
has been solved in transmission mode in Refs. 4�6 and an
estimate of $\alpha_\infty$, $\Lambda'$, and $\Lambda'$ made therein that gives a good
correlation between experience and theory. Porosity was not
estimated in this mode because of its weak sensitivity. Porosity
was estimated in reflected mode in our previous\\\\}}}

\parindent -1em
\includegraphics[width=0.95\linewidth]{g3.jpg}

{\footnotesize{FIG. 9. Variation of cost function U with the tortuosity for plastic foam M3.}}

\parindent 1em
\includegraphics[width=0.97\linewidth]{g2.jpg}
\parindent 0em
{\footnotesize{FIG. 10. Variation of cost function U with the porosity for plastic foam M3.}}
\\

\textnormal{\linespread{0.5}{\normalsize{studies$^{7,8,31,32}$ by solving the inverse problem at normal$^8$ and
oblique incidence$^31$ for a fixed value of tortuosity. Porosity
and tortuosity were measured$^7,32$ simultaneously for plastic
foams and air-saturated random packings of beads by measuring
reflected waves for each pair of incident angles.}}}

\parindent 1.5em

\textnormal{\linespread{0.5}{\normalsize{In this article, we determine the porosity and tortuosity
by solving the inverse problem for waves reflected by the
first interface, and by taking into account experimental data
for all measured incident angles.}}}

\parindent 1.5em

\textnormal{\linespread{0.5}{\normalsize{The inverse problem is to find values of parameters $\phi$
and $\alpha_\infty$ , which minimize the function}}}
\begin{small}
\begin{align}
	U''=F(\phi,\alpha_\infty)\sum_{\theta_{i}}\sum_{t_{i}}[p^r(x,\theta_i,t_i) \nonumber \\
	-r(\theta_{i},t_{i})*p^i(x,\theta_i,t_i)]^2 \tag{30}
\end{align}
\end{small}
\textnormal{\linespread{0.5}{\normalsize{where $p^r(x,\theta_i,t_i)$ represents the discrete set of values of the
experimental reflected signal for different incident angles $\theta_i$ ,$r(\theta_i,t_i)$ is the reflection coefficient at the first interface, and
$p^i(x,\theta_i,t_i)$ is the experimental incident signal. The term
$r(\theta_i,t_i)$*$p^i(x,\theta_i,t_i)$) represents the predicted reflected signal.
The inverse problem is solved numerically by the leastsquare
method.}}}

\parindent 1.5em

\textnormal{\linespread{0.5}{\normalsize{Experiments were performed in air with two broadband
Ultran NCT202 transducers with 190 kHz center frequency
in air and 6 dB bandwith that extends from 150 to 230 kHz.
An optical goniometer was used to position the transducers.
Pulses of 400 V were provided by a 5052PR Panametrics
puls receiver. The signals received were amplified to 90 dB
and filtered above 1 MHz to avoid high frequency noise.
Electronic interference was removed by 1000 acquisition averages.
The experimental setup is shown in Fig. 4. The duration
of the signal plays an important role$:$ its spectrum
must verify the condition of the high frequency approximation
referred to in Sec. IV.\\}}}

{\footnotesize{TABLE II. Reconstructed values of porosity and tortuosity by solving the inverse problem.}}
{\setlength{\extrarowheight}{3pt}
\begin{center}
	\begin{tabular}{cccc}
	\hline
	\hline
	\footnotesize{\qquad{Material}\qquad\qquad} & \footnotesize{M1\qquad\qquad} & \footnotesize{M2\qquad\qquad} & \footnotesize{M3\quad} \\ [0.5ex] 
	\hline 
	\footnotesize{\qquad{Tortuosity}\qquad\qquad} & \footnotesize{1.12\qquad\qquad} & \footnotesize{1.1\qquad\qquad} & \footnotesize{1.6\quad} \\ [-1ex]
	\footnotesize{\qquad{Porosity}\qquad\qquad} & \footnotesize{0.96\qquad\qquad} & \footnotesize{0.99\qquad\qquad} & \footnotesize{0.85\quad} \\ [0.5ex] 
	\hline
	\hline
	\end{tabular}
\end{center}


\end{document}
